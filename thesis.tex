% !TEX encoding = UTF-8

%% Davide Imola - VR386238
%% Esame di LaTeX

\documentclass[a4paper,titlepage]{book}

\usepackage[english,italian]{babel}
\usepackage{frontespizio}
\usepackage{amsmath,amssymb,graphicx}
\usepackage{natbib}
\usepackage{url}
\usepackage{lipsum}
\usepackage{hyphenat}
\usepackage{url}
\usepackage{listings}

\newcommand\abstractname{Abstract}  %%% here
\makeatletter
\if@titlepage
  \newenvironment{abstract}{%
      \titlepage
      \null\vfil
      \@beginparpenalty\@lowpenalty
      \begin{center}%
        \bfseries \abstractname
        \@endparpenalty\@M
      \end{center}}%
     {\par\vfil\null\endtitlepage}
\else
  \newenvironment{abstract}{%
      \if@twocolumn
        \section*{\abstractname}%
      \else
        \small
        \begin{center}%
          {\bfseries \abstractname\vspace{-.5em}\vspace{\z@}}%
        \end{center}%
        \quotation
      \fi}
      {\if@twocolumn\else\endquotation\fi}
\fi
\makeatother


\hyphenation{Ionic}
\hyphenation{Apache CouchDB}


\begin{document}

\pagestyle{plain}

%Generazione Frontespizio
\begin{frontespizio}
\Universita{Verona}
\Dipartimento{Informatica}
\Corso[Laurea]{Informatica}
\Annoaccademico{2016--2017}
\Titoletto{Tesi di Laurea Triennale}
\Titolo{Applicazione di Coaching in Ionic con interazione a DataBase NoSQL}
\Candidato[VR386238]{Davide Imola}
\Relatore{Prof. Graziano Pravadelli}
\Correlatore{Dott. Florenc Demrozi}
\end{frontespizio}

% Front
\frontmatter

% Generazione Sommario
\begin{abstract}
La presente tesi \`{e} una documentazione del codice relativo al progetto dell'applicazione per {\foreignlanguage{english} Smartphone} di {\foreignlanguage{english} coaching}. 

In questo documento si vedr\`{a} nel dettaglio la parte relativa all'applicativo scritto in Ionic 3 e l'interazione al {\foreignlanguage{english} DataBase NoSQL} Apache CouchDB.

Nei primi capitoli si vedranno quali componenti sono stati utilizzati per la strutturazione del progetto e delle semplici guide relative alla loro installazione e corretta configurazione.

Nei capitoli restanti andremmo a coprire invece diversi aspetti di studio e progettazione del codice, partendo dall'ideazione di una base per l'interfaccia grafica alla vera e propria stesura del codice.
\end{abstract}

% Generazione Indice
\tableofcontents

%Main
\mainmatter
% Capitolo 1
\chapter{Componenti utilizzate}


\section{Node.js}
Per prima cosa andiamo a osservare l'installazione di Node.js, piattaforma {\foreignlanguage{english} event-driven} per il motore JavaScript V8 di Chrome. Bisogna quindi andare a scaricare dal sito \url{https://nodejs.org/en/} la versione corrente.

Al termine dell'installazione sar\`{a} presente sul vostro sistema il {\foreignlanguage{english} node package manager} richiamabile da terminale con il comando
\begin{lstlisting}[language=bash]
  $ npm
\end{lstlisting}
che utilizzeremo successivamente.

\section{Ionic 3}
Per lo sviluppo della nostra applicazione si \`{e} utilizzato principalmente Ionic nella sua versione 3, un famoso e completo {\foreignlanguage{english} SDK open-source} per lo sviluppo di applicazioni ibride per dispositivi mobili. 
Ionic 3 si basa sull'uso di AngularJS e Apache Cordova e utilizza alcune tecnologie inerenti allo sviluppo {\foreignlanguage{english} Web} come CSS, HTML5 e Sass.

Per eseguire l'installazione basta semplicemente lanciare nel proprio terminale il seguente comando
\begin{lstlisting}[language=bash]
  $ npm install -g cordova ionic
\end{lstlisting}


\section{CouchDB}


\section{PostgreSQL}


\section{Python 3}


\section{Sorgenti}


\end{document}